\chapter{Introduction} 
\label{cap:ini}
\vspace{-2cm}


Water is essential for all human activities and agriculture is the largest freshwater consumer. Precipitation a phenomenon highly susceptible to variability, determines its availability\cite{calzadilla2013climate}. Research and apply accurate statistical models to forecast this phenomena has been acknowledged to play a key role for this sector of the human activity\cite{toth2000comparison}. Given the uncertainty and variability that drives its occurrence, it is recognised that is quite difficult to obtain reliable and accurate prediction models that can spatialy forecast this element of the hydrological cycle for short periods of time. \cite{brath1997role}.

The precipitation forecasting problem is commonly approached in different ways. The use of remote sensing observation with radars and satellite images addresses the issue based on the extrapolation of current weather condition, for very short term forecasting (scale of minutes). Unfortunately the use of radar and satellite images do not provide a satisfactory assessment of rain intensities in larger scales of time, in addition, using this technique in mountainous regions is difficult because of the occurrence of soil shading and the altitude effect\cite{toth2000comparison}. 

One other mean to obtain rainfall forecasting models is by time series analyses techniques. There are different approaches to time series forecasting, specially for climatic proposes.Traditionally forecasting has long been the domain of linear statistics, usual approaches to time series prediction, such as Box-Jenkins\cite{box1976time} or ARIMA (autoregressive integrated moving average) method\cite{pankratz1983forecasting}, considers that time series behaves as linear processes. Despite of its easy understanding and applicability it may be totally inappropriate to implement if the ongoing mechanism is subjected to an nonlinear processes \cite{zhang2003time}. 

In meteorology to deal with non linearity, it is generally used numerical weather prediction models (NWP) in applications such as Global Circulation Models (GCM). NWP is an initial-value problem for which initial data are not available in sufficient quantity and with sufficient accuracy, these models abstract some layers of information by discretising partial differential equations governing large scale atmospheric flow \cite{ghil1981applications}. This method can active acceptable accuracy in forecasting some meteorological phenomenas but when dealing with rainfall they yet have not active it \cite{ramirez2006linear}, mainly because of the physical complexity of precipitation processes and the reduced temporal and spacial scale involved in such phenomena that numerical models cannot resolve \cite{kuligowski1998localized}. One other drawback that NWP models such as GCM have is that they are computationally demanding and require powerful and expensive hardware to be implemented in a meteorological prediction center.

More recently researchers have been approaching such problem with artificial neural networks(ANN), this is a powerful alternative to traditional time-series modelling \cite{zhang1998linear} as for NWP models. ANNs are data-driven self adaptive methods that are able to understand and solve problems of which there's not enough data or observations to use more traditional statistical models\cite{zhang1998forecasting}, rainfall is such a phenomena and ANNs are suited and studied solution.

ANNs are a type of nonlinear model inspired by sophisticated functionalities of human brain. They are universal function approximators that can adaptively discover patterns from data, learn from experience and estimate any complex functional relationship with high accuracy\cite{zhang1998linear, wang2003artificial}, they mimics the brain functionalities both in knowledge acquisition through a learning process and memory by storing synaptic weights as acquired knowledge\cite{ferraudo}.

In the field of agriculture and applied math ANNs has been a successful tool to forecast meteorological indexes  \cite{kumarasiri2006rainfall, nasseri2008optimized, ramirez2006linear, luk2000study, french1992rainfall, toth2000comparison, partal2015daily}. In these studies the goal was to numerically predict, with a single ANN structure, the accumulative volume of precipitation in a given scale in a future period of time. The performance of these models were very correlated to the time scale of events that ANNs had to handle. In larger scale of time, such as months, the performance of ANNs are vastly superior then in shorter periods of time. This happens because in larger periods of time the probability of some precipitation be recorded is greater, consecutively models are not biased by a big number of observations with zero precipitation \cite{schoof2001downscaling} and in short scale of time rainfalls are dependent on small scale and unstable physical processes \cite{kuligowski1998localized}.

The objective of this paper is to create a methodology to predict the occurrence of rainfall. This is done by constraining the complexity of the predicted events by reducing the variance and rising the bias of the time series. To achieve this objective, a structure of artificial neural networks is being proposed which identifies the signs that lead to the occurrence of rain for each climatic season in short periods of time, letting the ANNs to predict whether or not it is going to rain. The proposed model is intended to filter which days are propitious to rain, so that only the climate variables in the periods that lead to rain are used in quantitative models. With this technique quantitative models can improve its forecasting performance in shorter periods of time and consecutively becoming computationally lighter by reducing the volume of data used in the training stage of the models. 





