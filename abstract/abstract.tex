% $Id: abstract.tex,v 1.1 2003/04/10 23:12:59 gweber Exp $

%%%%%%%%%%%%%%%%%%%%%%%%%%%%%%%%%%%%%
%% Abstract
%% Copyright 2003 Dehon Charles Regis Nogueira.
%% Este documento é distribuído nos termos da licença
%% descrita no arquivo LICENCA que o acompanha.
%%%%%%%%%%%%%%%%%%%%%%%%%%%%%%%%%%%%%


\begin{abstract}
\onehalfspacing  % espaçamento 1,5
\vspace*{-0.5cm}
\noindent
Precipitation, in short periods of time, is a phenomenon associated with high levels of uncertainty and variability. Given its nature, traditional forecasting techniques are expensive and computationally demanding. This paper presents a soft computing technique to forecast the occurrence of rainfall in short ranges of time by Artificial Neural Networks(ANNs) in accumulated periods from 3 to 7 days for each climatic season, mitigating the necessity of predicting its amount. With this premise it is intended to reduce the variance, rise the bias of data and lower the responsibility of the model acting as a filter for quantitative models by removing subsequent occurrences of zeros values of rainfall which leads to bias the and reduces its performance. The model were developed with time series from  10 agriculturally relevant regions in Brazil, these places are the ones with the longest available weather time series and and more deficient in accurate climate predictions, it was available 60 years of daily mean air temperature and accumulated precipitation which were used to estimate the potential evapotranspiration and water balance; these were the variables used as inputs for the ANNs models. The mean accuracy of the model for all the accumulated periods were 78\% on summer, 71\% on winter 62\% on spring and 56\% on autumn, it was identified that the effect of continentality, the effect of altitude and the volume of normal precipitation, have an direct impact on the accuracy of the ANNs. The models have peak performance in well defined seasons, but looses its accuracy in transitional seasons and places under influence of macro-climatic and mesoclimatic effects, which indicates that this technique can be used to indicate the eminence of rainfall with some limitations

\vspace{1.2cm}
\noindent
\textbf{Keywords:} artificial neural networks. rainfall forecasting. multilayer perceptron


\end{abstract}



