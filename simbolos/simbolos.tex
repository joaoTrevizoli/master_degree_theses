\chapter*{Lista de Símbolos}

\vspace*{-0.5cm}

\onehalfspacing

% Comentario :
% Os simbolos deven ser colcoados em ordem alfavetica en função da sua definição

\noindent
\begin{tabular}{l c p{.85\linewidth}}
$\theta_i$             & & Ângulo de fase na barra $i$\\

$g_{ij}$               & & Condutância da linha no ramo $ij$ \\

$Y$                    & & Conjunto das linhas que podem ou não serem adicionadas no ramo $ij$\\

$\Omega_b$             & & Conjunto de barras \\

$\Omega_l^1$           & & Conjunto de caminhos nos quais existem Linhas na configuração base\\

$\Omega_l^2$           & & Conjunto de caminhos novos (onde serão adicionadas novos Linhas)\\

$\Omega_l^0$           & & Conjunto de linhas existentes na configuração base\\

$\Omega_l$             & & Conjunto de ramos\\

$c^n_{ij}$             & & Custo de construção das linhas no ramo $ij$\\ 

$d_{i}$                & & Demanda na barra $i$\\

$\epsilon_f$           & & Error da condição de factibilidade \\

$\epsilon_o$           & & Error da condição de otimalidade\\

$\epsilon_{\mu}$       & & Error do parâmetro de barreira \\

$\gamma$               & & Fator de segurança \\

$\overline{f}_{ij}^0$  & & Fluxo de potência ativa máximo nos ramos para o conjunto de linhas já existentes\\

$\overline{f}_{ij}^1$  & & Fluxo de potência ativa máximo nos ramos para o conjunto de linhas já existentes ou linhas adicionadas em paralelo\\ 

$\overline{f}_{ij}^2$  & & Fluxo de potência ativa máximo nos ramos para o conjunto de linhas correspondentes aos novos caminhos\\ 

$\overline{f}_{ij}$    & & Fluxo de potência ativa máximo permitida no ramo $ij$ para linhas novas\\

$f_{ij}^0$             & & Fluxo de potência ativa nos ramos para o conjunto de linhas já existentes\\

$f_{ij}^1$             & & Fluxo de potência ativa nos ramos para o conjunto de linhas já existentes ou linhas adicionadas em paralelo\\

$f_{ij}^2$             & & Fluxo de potência ativa nos ramos do conjunto de linhas correspondentes aos novos caminhos\\

$f_{ij}$               & & Fluxo de potência ativa no ramo $ij$ para linhas novas\\

$f_{ij,y}$             & & Fluxo na linha $y$ do ramo $ij$\\

$p_{i}$                & & Geração na barra $i$\\

$\overline{p}_{i}$     & & Geração máxima na barra $i$\\

$v$                    & & Investimento devido às adições de Linhas no sistema - Função Objetivo\\

$ij$                   & & Linha entre as barras $i$ e $j$ \\

$n_{ij}$               & & Número de linhas adicionadas no ramo $ij$


\end{tabular}

\newpage
\noindent
\begin{tabular}{l c p{.85\linewidth}}

$\overline{n}_{ij}^2$  & & Número máximo de linhas em caminhos novos \\

$\overline{n}_{ij}^1$  & & Número máximo de linhas que podem ser adicionadas em paralelo às linhas dos caminhos já existentes\\

$\overline{n}_{ij}$    & & Número máximo de Linhas que podem ser adicionados no ramo $ij$\\

$n_{ij}^1$             & & Número de linhas adicionadas em paralelo às linhas já existentes\\

$n_{ij}^{0}$           & & Número de linhas existentes na configuração base no ramo $ij$\\

$n_{ij}^2$             & & Número de linhas novas adicionadas no ramo $ij$\\

$\gamma_{ij}$          & & Susceptância nas linhas do ramo $ij$\\

$\gamma_{ij}^0$        & & Susceptância nas linhas existente do ramo $ij$\\

$w_{ij,y}$             & & Variável binária correspondente à linha $y$ candidata a ser adicionada ou não no ramo $ij$\\

$x_{ij}$               & & reatância do circuíto $ij$\\

$q_{i}$                & & vetor de geração de potência reativa na barra $i$\\

$\overline{q_i}$       & & limite máximo de geração de potência reativa na barra $i$\\

$\underline{q_i}$      & & limite mínimo de geração de potência reativa na barra $i$\\

$e_{i}$                & & vetor de demanda de potência reativa na barra $i$\\

$V_{i}$                & & magnitude de tensão na barra $i$\\

$\overline{V_{i}}$     & & limite máximo da magnitude de tensão na barra $i$\\

$\underline{V_{i}}$    & & limite mínimo da magnitude de tensão na barra $i$\\

$e_{i}$                & & vetor de demanda de potência reativa na barra $i$\\

$s_{ij}^{de}$          & & fluxo de potência aparente (MVA) no ramo $ij$ saindo do terminal\\

$s_{ij}^{para}$        & & fluxo de potência aparente (MVA) no ramo $ij$ chegando no terminal\\

$\overline{s_{ij}}$    & & limite de fluxo de potência aparente (MVA) no ramo $ij$\\

$\theta{ij}$           & & diferença angular entre as barra $i$ e $j$\\

$\Omega_{bi}$          & & conjunto das barras vizinhas da barra $I$\\

$g_{ij}$               & & condutância da linha no ramo $ij$\\

$g_{ij}^0$             & & condutância existente da linha no ramo $ij$\\

$b_{ij}$               & & susceptância da linha no ramo $ij$\\

$b_{ij}^{sh}$          & & susceptância shunt da linha no ramo $ij$\\

$b_{i}^{sh}$           & & susceptância shunt na barra $i$\\

$G_{ij}$               & & matriz de condutância\\

$B_{ij}$               & & matriz de susceptância\\

\end{tabular}
