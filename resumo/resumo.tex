% $Id: resumo.tex,v 1.1 2003/04/10 23:12:59 gweber Exp $
%%%%%%%%%%%%%%%%%%%%%%%%%%%%%%%%%%%%%
%% Resumo	
%% Copyright 2003 Dehon Charles Regis Nogueira.
%% Este documento é distribuído nos termos da licença
%% descrita no arquivo LICENCA que o acompanha.
%%%%%%%%%%%%%%%%%%%%%%%%%%%%%%%%%%%%%

\begin{resumo} 
\onehalfspacing  % espaçamento 1,5
\vspace*{-0.5cm}
\noindent
Neste trabalho foi desenvolvido uma interface gráfica para uso via web, utilizando-se as linguagens shell-script e PHP, com o objetivo de facilitar a configuração e monitoração de diferentes serviços necessários em um servidor de rede, tais como:  firewall, DHCP, squid/proxy, DNS, e-mail, dentre outros. Para isso, utilizou-se uma estratégia de desenvolvimento modular, para facilidade de uso e que permite a inclusão de novos módulos posteriormente. A ferramenta foi totalmente desenvolvida com software livre e o acesso ao seu código permite alterações de acordo com as necessidades do usuário.

\vspace{1.2cm}
\noindent
\textbf{Palavras-chave:} Servidores. Redes. Firewall. Segurança.
\end{resumo}


