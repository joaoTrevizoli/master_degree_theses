\chapter{CONCLUSION}
\label{cap:cap4}
\vspace{-2cm}

The objective of this dissertation was to develop an automatic ANN modelling strategy that through the analysis of time series predicts the occurrence of rainfall greater than 5mm for each climatic season for accumulated periods from 3 to 7 days. The study has led to the conclusion that the ANNs can forecast these events with an average accuracy for all the accumulated periods of 78\%  on summer,  71\% on winter, 62\% on spring  and 56\% on autumn. Despite the results, the performance of these models could be improved in future studies, by using training algorithms that are capable of converging on results closer to the global optimum such as training feedforward ANNs with genetic algorithms.

Macroclimatic and mesoclimatic effects, such as the effect of continentality and the effect of altitude as well as the normal precipitation volume, has an direct impact on the forecasting accuracy of the ANNs in well defined seasons. Furthermore despite the relatively lower forecasting performance of transitional seasons, the most important seasons for Brazilian crop production are the summer and winter that are those that the model had best accuracy, nevertheless the results of autumn and spring are still applicable with some limitations. To improve this technic different classificatory algorithms could be implemented, in addition exploratory multivariate statistical procedures, such as principal component analysis or correspondence analysis, would better select input variables. This work can also be applied and improved on other fields of human activities such as transportation and traffic control and being employed to create alert systems given its binary nature. However this type of ANNs structures are suited as an indicative of rainfall eminence and in future studies separate models can be developed to forecast its volume. 
